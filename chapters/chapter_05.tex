\chapter{Chosen technologies}\label{technologies}

	This chapter describes pros and cons of available tools, frameworks and architectural patterns, that may be used and
	explains why and which technologies I finally selected.

	\section{REST vs. SOAP}
	
	Although implementing a RESTful API is one of the main topics of this master's thesis, it is good to compare other
	possibilities too. Currently the most commonly used technology to build SOA in addition to REST is SOAP. So what are
	the differences between them? Why should not I use SOAP, when thousands of enterprise systems are using it?
	
	Sometimes people talk about SOAP like it was something that is deprecated, or even dead and REST is its successor and
	is much better and modern. This is not true and not even part of it. REST is no revolution in SOA, but rather
	an evolution. SOAP has its place when it comes to a question of implementing services or APIs and so does REST. The
	main difference is: \textbf{SOAP} is aimed for \textbf{server-server} communication and \textbf{REST} is more suitable
	for \textbf{client-server} communication.
	
	Let's forget about the RESTful API requirement and make it just an API. What are the pros and cons of REST, resp. SOAP
	if I could choose one or the another on my own? The only thing that I have to keep in mind is: I'm implementing an API
	for admission processing and I have two different API consumers.
	
	\section{BPEL vs. BPMN}
	
	\section{JAX-RS implementation}
	
	\section{Spring Core, MVC, Security, \ldots}
	
	\section{Spring Roo}