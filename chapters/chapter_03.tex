\chapter{BPM and jBPM}\label{bpm}

	\cite{bpm}
	Business process modeling (BPM), sometimes called business process management, refers to the design and execution of
	business processes.
	
	It does not have to be necessarily used in a context of \gls{IT} and software development. Primary field of BPM falls
	into management, though. Before \gls{ICT} was widely spread and automatic software processing was a dream, BPM was
	manual and paper driven.

	But yes, BPM is also closely aligned with the notion of \gls{SOA}, particularly the emerging W3C web services
	stack. Whereas the traditional use of a workflow was about the movement of work from person to person within an
	organization, contemporary BPM processes are built to interact as services with other systems, or even to orchestrate
	or choreograph other systems, including the business processes of other companies.

	\section{Business Process}

	A business process is a service, one intended to be called by other systems, and these calls drive its execution.
	Realizing this fact is one of the first big steps in understanding BPM.
	
	Being algorithmic, a process can potentially be run by some sort of process engine. As long as the process can be
	expressed in a form that is syntactically and semantically unambiguous that is, in a programming language or other
	interpretable form the engine can accept it as input, set it in motion, and drive its flow of control. To be precise,
	the engine creates and runs instances of a given process definition. The steps of the process are called activities or
	tasks.
	
	\section{Important process modelling terms}
	\begin{itemize}
		\item \textbf{Process definition}
		The basic algorithm or behavior of the process.
		\item \textbf{Process instance}
		An occurrence of a process for specific input. Each instance of the travel reservation process, for example, is tied
		to a specific customer's itinerary.
		\item \textbf{Activity or task}
		A step in a process, such as sending a flight request to the airline.
		\item \textbf{Automated activity or automated task}
		A step in a process that is performed directly by the execution engine.
		\item \textbf{Manual activity or manual task}
		A step in a process that is meant to be performed by a human process participant. 
	\end{itemize}
	
	The distinction between manual and automated activities is extremely important. At one time, before the reign of
	software, a business process was completely manual and paper-driven: paper was passed from person to person, and was
	often misplaced or delayed along the way. Now, much of the process runs on autopilot.

	Automated activities generally fall into two categories:
	\begin{itemize}
		\item \textbf{Interactions with external systems} e.g., sending a booking request to an airline
		\item \textbf{Arbitrary programmatic logic} e.g., calculating the priority of a manual task	   
	\end{itemize}
