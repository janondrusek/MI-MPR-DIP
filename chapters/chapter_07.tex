\chapter{Conclusion}\label{cha:conclusion}

	Purpose of this chapter is to summarize everything that has been done in a manner of this master's thesis and to
	describe results of RESTful API team's work.
	
	Results are evaluated against primary tasks and Catalogue~of~requirements~\ref{sec:catalogue}.
	
	\section{RESTful API and its application}
	
	Functional requirements and special requirements of the other two teams practically describe public interface of
	RESTful API.
	Positive result is that all points of Functional requirements are fulfilled.
	
	What I was unable to achieve is to implement Admission filters. Lower priority of this special requirement by Web UI
	team caused its last place in the implementation plan, which unfortunately was not completely finished. This however
	has no critical impact on RESTful API's functionality.
		
	\section{jBPM}
	
	The \gls{BPM} part of this project was something very new to me and I had to dive into business processes in general
	first. jBPM as a processing machine was studied afterwards. It indeed is a technology, which is worth working with
	and it found its use case in RESTful API application. I will consider jBPM in my further projects, where it may be
	helpful, as well.
	
	We managed to create business processes for all the Conditions for admission and Dean's directive for admission
	process. Although they work very well when tested isolated, due to lack of time we did not manage to hook BPM processes
	with RESTful API's service layers. They both work with the same database and data, unfortunately do not play well
	together yet.
	
	\section{Open questions}
	
	What shall be done with jBPM part? If RESTful API should be deployed into production, code change and some development
	is required. This should be a simple task, because both API and jBPM work correctly when isolated.
	
	If Conditions for admission or Dean's directive change, how does it effect BPM models? It is quite simple to perform
	model changes using GUI tools. They however have to be deployed into RESTful API. Currently BPM models are stored as
	plain files. An enhancement would be to store them encoded in the database and swap them via some simple GUI, e.g.
	custom web application.

	\section{Lessons learned}

	Each project of such scale should teach all involved people a lesson and all of them should gain valuable experience.
	RESTful API is not an exception.
	
	The main problem was time and underestimating of the whole project. It turns out, that the required amount of time was
	120 man days for two skilled developers. This is three months in two people working on the project full time!
	
	Another problem was lack of experience in selected technologies. None of the two working on RESTful API was experienced
	in BPM and jBPM, and David was lacking Java, Spring and JPA skills as well. This, of course, slowed down the
	development.
	
	Assigning one more person to a project of such scale should be considered next time.

	\section{Contribution to the community}
	
	During the development of RESTful API and possibly some other school projects, I discovered, that all of my Maven based
	applications share lots of common dependencies, plugins and settings. This is why I stared a tiny project simply called
	\textit{parent}. It contains lots of version properties and common settings. Can be simply used as a Maven project
	parent.
	
	Another one is a tiny set of JUnit and EasyMock testing templates. One has to follow a pattern, where a single class is
	tested and all of its dependencies can be injected and are mocked. This is practically what a developer has to do
	manually for each test class. My project called \textit{test-templates} does this automatically just by using
	annotations. This saves lots of typing.
	
	Both projects can be found on my~GitHub~\cite{github} and are public.
