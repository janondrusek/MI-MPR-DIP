\chapter{RESTful API with JAX-RS}\label{rest}

	Nowadays, Internet consumers demand fast growth of various services and integration of their favourite ones. As en
	example I can point out synchronization of contact list between very popular social networks, e-mail providers and
	phone contact lists. 
	
	Other example may be growing amount of \verb|mashups|\footnote{Applications that are created via
	combination of multiple different services. Such application, almost exclusively web based, can be created very quickly
	by consuming several \gls{API}s. Not necessarily from the same provider.} and uncountable number of
	\verb|startups|\footnote{Constantly rising amount of web applications that focus on fast growth of attracted users.
	They offer various services, which are often very innovative and experimental. One successful example is popular
	social network and my favorite information channel - Twitter.}, who often provide RESTful or different type of public
	\gls{API}.

	\section{Talking about REST, what is it?}
	
	\gls{REST} or RESTful programming is not defined by any official standard and there are no official guidelines or rules
	for it.
	So what is it then? It is and architectural and programming style for Web, where a set of constraints is defined. Lots
	of text has been written about it during past years and describing the whole idea of REST is out of scope of this
	master's thesis. I can however try to point out the most significant, important and basically, what I personally
	managed to adopt.
	
	\subsection{Main principles of REST, RESTful web service}
	
	There are several architectural principles that one should keep in mind when thinking of REST \cite[p.~3]{restful}:
	
	\begin{itemize}
	  	\item \textbf{Addressable resources} 
	  	The key abstraction of information and data in REST is a resource, and each resource must be addressable via a
	  	\gls{URI}.
		\item \textbf{A uniform, constrained interface}
		Use a small set of well-defined methods to manipulate your resources.
		\item \textbf{Representation-oriented}
		You interact with services using representations of that service. A resource referenced by one URI can have different
		formats. Different platforms need different formats. For example, browsers need HTML, JavaScript needs JSON (JavaScript
		Object Notation), and a Java application may need XML.
		\item \textbf{Communicate statelessly}
		Stateless applications are easier to scale.
		\item \textbf{\gls{HATEOAS}}
		Let your data formats drive state transitions in your applications.
	\end{itemize}
	
	\gls{HATEOAS} is often understood as a core principle of \gls{REST}. It carries an idea of resource representation via
	links and stateless implementation of services.
	
	RESTful web services are the result of applying these constraints to services that utilize web standards such as
	\gls{URI}s, \gls{HTTP}, \gls{XML}, and \gls{JSON}.
	
	\subsection{Back to the roots, HTTP is reborn}
	
	\gls{SOA} has been in this world for a long time. Many different approaches and technologies exist to implement it.
	From those worth to mention: DCE, CORBA, Java RMI, \ldots They offer robust standards, one can build large, complex
	and scalable systems on top of it, but there is a cost. They often bring huge complexity and maintenance requirements
	into place.
	
	Currently, when one says \gls{SOA}, it often evokes \gls{SOAP} in a mind that spent several years using technologies
	mentioned above. This however is not a bad thing. \gls{SOAP} is used very widely and is perfectly suitable for
	developing services and \gls{API}s. But it is definitely not a lightweight technology and it is not ideal for
	everything. Its most common use case is for server-server communication in enterprise systems.
	
	Nowadays, we need something quickly adoptable, widely spreadable, platform and technology independent and client
	oriented. This needs a completely different approach and new way of thinking when it comes to \gls{SOA}. It is about
	Web, so why not to start with something that is Web, as we see it today, based on? Yes, it is \gls{HTTP}.
	
	Although \gls{REST} is not protocol specific, when saying \gls{REST} it usually automatically means \gls{REST} +
	\gls{HTTP}. No wonder. \gls{HTTP} is perfectly suitable for client-server \gls{SOA}, it is just about the way of
	thinking. It offers transport layer, request-response mechanism, descriptive responses, caching mechanism and many
	more. It is true that in past years, when various types of web applications started to appear, many web developers
	limited their thinking and use of \gls{HTTP} to two basic cases:
	
	\begin{itemize}
	  \item GET a page with \gls{URI}, perhaps containing a few query parameters
	  \item POST a form
	\end{itemize}
	
	\begin{program}
	\caption{HTTP GET request/response example of a standard web page}\label{http_get_web}
	\begin{verbatim}
	GET /index.html HTTP/1.1
	User-Agent: curl/7.24.0 (i686-pc-cygwin) ...
	Host: www.google.sk
	Accept: text/html,application/xhtml+xml,application/xml;q=0.9,*/*;q=0.8
	Accept-Language: sk-sk,en;q=0.5

	HTTP/1.1 200 OK
	Date: Thu, 07 Jun 2012 11:25:15 GMT
	Expires: -1
	Cache-Control: private, max-age=0
	Content-Type: text/html; charset=ISO-8859-2
	Set-Cookie: ... expires= ...; path=/; domain=.google.sk
	Set-Cookie: ... expires= ...; path=/; domain=.google.sk; HttpOnly
	Server: gws
	X-XSS-Protection: 1; mode=block
	X-Frame-Options: SAMEORIGIN
	Transfer-Encoding: chunked

	<!doctype html><html ...><head> ... <body> ...
	\end{verbatim}
	\end{program}
	
	The example above \ref{http_get_web} shows most common HTTP request and response, when browsing the web via standard
	web browser. It requests object \textbf{/index.html} using \textbf{GET} method placed on host \textbf{www.google.sk}.
	My client also put several HTTP headers into the request:
	
	\begin{itemize}
	  \item \textbf{Accept:} text/html,application/xhtml+xml,application/xml;q=0.9,*/*;q=0.8
	  \item \textbf{Accept-Language:} sk-sk,en;q=0.5
	\end{itemize}
	
	Also the request does not contain any request body, as it is GETting information from the server. 
	
	The response of the message received is 200, which means OK - success. An overview of all available HTTP response codes
	can be found on-line at \cite{httpcodes}. \textbf{Content-type} header of the response message says that the body
	received is of type HTML.
	
	RESTful web service needs more than that and luckily \gls{HTTP} offers much more. It will be better to point out its
	features in a relationship to each of REST's architectural principles.
	
	\subsection{Addressable resources, URIs and links}
	
	Each \textbf{resource} in a system should be reachable through a \textbf{unique identifier}. When reflected to the idea
	of REST and HTTP, URIs will automatically come to mind. The format of a URI looks like this \cite{uri}:
	
	\begin{verbatim}
	<scheme>://<authority><path>?<query>
	\end{verbatim}
	
	The above is a citation directly from RFC, but for purposes of this master's thesis should be rewritten into more
	detailed form:
	
	\begin{verbatim}
	<scheme>://<host>:<port>/<path>?<queryString>#<fragment>
	\end{verbatim}
	
	Where in the RESTful world these parts usually mean:
	\begin{itemize}
	  \item \textbf{scheme} typically \textbf{http} or \textbf{https}
	  \item \textbf{host} aka server name, e.g. \textbf{fit.cvut.cz} and \textbf{port}
	  \item \textbf{path} to the resource on server, e.g. \textbf{/admission/123-456-01}
	  \item \textbf{queryString} after the \textbf{?} is typically used for a set of resources and can be a page number,
	  number of items in the set, or a filter definition and many more, e.g. \textbf{?page=1\&limit=10}
	  \item \textbf{fragment} after the \textbf{\#} usually points to a certain place in a document
	\end{itemize}
	
	An example of such URI pointing to a set of resources may be:
	
	\begin{verbatim}
	http://pririz.is.fit.cvut.cz:9090/admission/services/admission
		?page=2&count=20
	\end{verbatim}
	 
	An example of URI pointing to a concrete single resource:
	
	\begin{verbatim}
	http://pririz.is.fit.cvut.cz:9090/admission/services/admission
	/123-456-01
	\end{verbatim}
	
	Characters allowed in \gls{URI} string are all alphanumeric, comma, dash, asterisk and underscore. Space is converted
	into plus and other characters are encoded using specific schema into a two digit hexadecimal number, which is appended
	to the \% character.
	
	\subsection{The uniform, constrained interface, HTTP methods}
	
	This is probably the prettiest part of a relationship between REST and HTTP. It may be a bit difficult to adopt this
	principle for a person, who spent a couple of years developing CORBA or SOAP services.  There is a finite set of HTTP
	methods and all REST operations have to stick to it. All other parameters describing operations must be omitted from
	\gls{URI}.

	Let's see what \gls{HTTP} offers:
	
	\begin{table}[h]\centering
	 	\begin{minipage}{12.9cm}
		\begin{tabular}{l|c|c|p{6cm}}
		\hline
		Method & \verb|Idempotent|\footnote{Idempotent means that no matter how many times you apply the operation, 
		the result is always the same.} & \verb|Safe|\footnote{Safe means that invoking an operation does not change the 
		state of the server at all. This means that, other than request load, the operation will not affect the server.} 
		& Operation(s)\\
		\hline
		GET & yes & yes & read - query information from a server\\
		POST & no & no & write, update - both can change a server state in a unique way each time executed\\
		PUT & yes & no & update for the known resource, updating the same resource more than once does not effect it\\
		DELETE & yes & no & delete - removes resource\\
		HEAD & yes & yes & read without response body, returns only response headers\\
		OPTIONS & yes & yes & information about communication options with server\\
		\end{tabular}
	    \renewcommand{\footnoterule}{}
	    \end{minipage}
	\caption{An overview of HTTP methods and their roles in a RESTful service}
	\label{http_methods}
	\end{table}
	
	HTTP contains a few other methods (TRACE, CONNECT), which are unimportant for purposes of RESTful services.
	
	What is more interesting, nowadays a couple of non-HTTP methods have appeared, which may be good for RESTful service
	design in the future. Namely PATCH (very similar to the PATCH method found in \verb|WebDAV|\footnote{WebDAV stands for
	Web-based Distributed Authoring and Versioning. It is a set of extensions to the HTTP protocol which allows users to 
	collaboratively edit and manage files on remote web servers.}) and MERGE. According to various sources I have found on
	the web, namely \cite{httppatch}, \cite{msdnpatchmerge} or \cite{rfc2616}, they may appear in further HTTP
	specifications. They are not a part of current HTTP 1.1.
	
	PATCH and MERGE are used for partial update of known resources that contain large amount od data and updating the whole
	object would be a lengthy and ineffective operation. This however can be simulated using POST and specifying detailed
	path of the resource via URI.
	
	\subsection{Representation-oriented}
	
	I already described that each resource has its own URI and client-server principle using HTTP. Its methods allow the
	client to receive current representation via GET method, remove it from server via DELETE or change the representation
	via POST and PUT methods. Concrete representation can be received in JSON, XML, YAML or any other format one can
	imagine.
	
	Representation format is agreed between client and server in a RESTful system interaction. HTTP offers such feature
	by specifying \textbf{Content-Type} header. Its value string is represented by \gls{MIME} format:
	
	\begin{verbatim}
	<type>/<subtype>[;name=value;name=value...]
	\end{verbatim}
	
	An example may be:
	
	\begin{verbatim}
	text/html; charset=utf-8
	text/xml
	text/json
	application/xml
	application/json
	\end{verbatim}
	
	To choose preferred format(s), client can specify \textbf{Accept} HTTP header in a request. Now it becomes more clear
	how REST and HTTP perfectly fit each other. Together they offer addresability, method choice and object representation
	format.
	
	\subsection{Communicate statelessly, (no)sessions}
	
	HTTP offers powerful client session management, which is commonly used when browsing the web via traditional web
	browser. It stores so called \textbf{Cookies} when a server asks for it in response headers, which are then sent back
	to the server with subsequent requests. This is how the server handles stateful interaction with the client over HTTP.
	
	Stateless communication in REST means that there is no client session data stored on a server. In other words, none of
	the above is performed. It does not mean that RESTful application cannot be stateful, though.
	
	Reason for this is simplicity, which further leads to easily scalable RESTful service. It is generally much less
	difficult to build a cluster of stateless applications than to handle session replication and possibly another service
	layer.
	
	\subsection{HATEOAS}
	
	\cite[p.~11]{restful} Hypermedia is a document-centric approach with the added support for embedding links to other
	services and information within that document format.
	
	There are several ways how to understand any apply this RESTful architectural principle. One use case is to add
	hyperlinks when composing complex and large objects. This avoids unexpected server load, delay in response to the
	client and helps to reference dependent or embedded objects without bloating the response.
	
	An example of server response without any hyperlinks:
	
	\begin{lstlisting}[tabsize=2]
	<terms>
		<term>
			...
			<registrations>
				<registration>
					<admission>
						<code>96858805</code>
						<type>P</type>
						<accepted>false</accepted>
						... some huge object containing 
						a lot of information
					</admission>
					<term>
						<dateOfTerm>2012-05-10...</dateOfTerm>
						<room>BS</room>
						<capacity>1500</capacity>
						<registerFrom>2012-05-03...</registerFrom>
						<registerTo>2012-05-08...</registerTo>
						<apologyTo>2012-05-08...</apologyTo>
						... another huge object containing 
						a lot of information, graph can be even circular
					</term>
				</registration>
				...
			</registrations>
			...
		</term>
		...
	</terms>
	\end{lstlisting}
	
	Let's apply embedded hyperlinks on the document above:
	
	\begin{lstlisting}[tabsize=2]
	<terms>
		<term>
			...
			<registrations>
				<registration>
					<admission>
						<link href="http://.../admission/96858805" 
						method="GET" rel="admission" />
					</admission>
					<term>
						<link href="http://.../term
						/dateOfTerm:2012-05-10T14:22:00/room:BS"
							method="GET" rel="term" />
					</term>
				</registration>
				...
			</registrations>
			...
		</term>
		...
	</terms>
	\end{lstlisting}
	
	This concept of HATEOAS is called aggregation. But it isn't everything. In a case that the server would return
	thousands of term objects and each of them would contain thousands of registrations including admissions, the response
	would be, again, very large. However, there is one even more interesting part of HATEOAS - the \uv{engine}.
	
	Core of the engine principle is not to return the whole set of object available, but just a subset of it and to
	tell the client, where to find the rest:
	
	\begin{lstlisting}[tabsize=2]
	<terms>
		<count>5</count>
		<totalCount>123</totalCount>
		<link href="http://.../term?page=3&count=5" method="GET" 
			rel="next" />
		<link href="http://.../term?page=1&count=5" method="GET" 
			rel="previous" />
		<term>
			...
			<registrations>
				<registration>
					<admission>
						<accepted>false</accepted>
						<link href="http://.../admission/96858805" 
							method="GET" rel="admission" />
					</admission>
					<term>
						<link href="http://.../term
						/dateOfTerm:2012-05-10T14:22:00/room:BS"
							method="GET" rel="term" />
					</term>
				</registration>
				...
			</registrations>
			...
		</term>
		...
	</terms>
	\end{lstlisting}
	
	This approach saves a lot of server resources and prevents client from unexpected delays due to large responses. Such
	response should be always returned in constant time, because the request query defines its maximum size - number of
	objects.
	
	\section{REST, Java and JAX-RS}
	
	\cite[p.~xiii]{restful} The \gls{JAXRS} is a new API that aims to make development of RESTful web services in Java
	simple and intuitive. The initial impetus for the API came from the observation that existing Java Web APIs were generally either:
	\begin{itemize}
	  \item Very low-level, leaving the developer to do a lot of repetitive and error-prone work such as URI parsing and
	  content negotiation, or
	  \item Rather high-level and proscriptive, making it easy to build services that conform to a particular pattern but
	  lacking the necessary flexibility to tackle more general problems. 
	\end{itemize}

	JAX-RS is one of the latest generations of Java APIs that make use of Java annotations to reduce the need for standard
	base classes, implementing required interfaces, and out-of-band configuration files. Annotations are used to route
	client requests to matching Java class methods and declaratively map request data to the parameters of those methods.
	Annotations are also used to provide static metadata to create responses.
	
	JAX-RS also provides more traditional classes and interfaces for dynamic access to request data and for customizing
	responses.
	
	This is a brief description of what JAX-RS is. Its usage and development approach will be demonstrated in following
	parts of this master's thesis.
